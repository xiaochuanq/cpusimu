\documentclass[12pt,twoside]{article}

%================================= Packages ==================================%
\usepackage{fullpage}
\usepackage{graphicx}
\usepackage{amsfonts}
\usepackage{amssymb}
\usepackage{amsmath}
\usepackage{latexsym}
\usepackage{enumerate}
\usepackage{hyperref}
\usepackage{algorithmic}
\usepackage{listings}
\usepackage{multirow}
\usepackage[usenames,dvipsnames]{color}

%================================== Banner ===================================%

\newcommand{\name}{Ben Charrow}
\newcommand{\class}{CIS 501 -- Architecture}
\newcommand{\problemset}{Specify a problemset!}
\newcommand{\duedate}{}
\newcommand{\collab}{}
\newcommand{\banner}{
\begin{center}
  \framebox{
    \vbox{
      {\bf\large\class} \\ \vfill
      {\bf\large\problemset} \\
      \name\;\collab \hfill  \hfill \duedate \\
    }
  }
\end{center}
\bigskip
}

%=============================== Math commands ===============================%
% Stuff for probability
\newcommand{\prob}[1]{\ensuremath{\mathbf{Pr}\left[#1\right]}}
\newcommand{\expec}[1]{\ensuremath{\mathbf{E}\left[#1\right]}}
\newcommand{\var}[1]{\ensuremath{\operatorname{Var}\left[#1\right]}}

% Custom functions
\DeclareMathOperator*{\argmax}{arg\,max}
\DeclareMathOperator*{\IG}{IG}
\newcommand{\tn}[1]{\textnormal{#1}}
% Braces!
\newcommand{\curly}[1]{\left(#1\right)}

%================================== Graphs ===================================%

\newcommand{\graph}[2]{
\subsection{#1}
\begin{figure}[h]
  \centering
  \scalebox{1.7}{\includegraphics{#2}}
\end{figure}
\newpage
}
%=================================== Code ====================================%
\definecolor{MyDarkGreen}{rgb}{0.0,0.4,0.0}

\lstset{language=C,                             % Use C
        frame=single,                           % Single frame around code
        basicstyle=\scriptsize\ttfamily,        % Use small true type font
        keywordstyle=[1]\color{Blue}\bf,        % C functions bold and blue
        keywordstyle=[2]\color{Purple},         % C function arguments purple
        keywordstyle=[3]\color{Blue}\underbar,  % User functions underlined and blue
        identifierstyle=,                       % Nothing special about identifiers
                                                % Comments small dark green courier
        commentstyle=\usefont{T1}{pcr}{m}{sl}\color{MyDarkGreen}\scriptsize,
        stringstyle=\color{Purple},             % Strings are purple
        showstringspaces=false,                 % Don't put marks in string spaces
        tabsize=4,                              % 5 spaces per tab
        numbers=left,                           % Line numbers on left
        firstnumber=1,                          % Line numbers start with line 1
        numberstyle=\scriptsize\color{Blue},    % Line numbers are blue
        stepnumber=5,                           % Line numbers go in steps of 5
        basicstyle=\tiny
        }

\renewcommand{\name}{Ben Charrow, Congyun Gu, Divya Kali Ranga, Xiaochuan Qin}
\renewcommand{\problemset}{Homework 6}
\renewcommand{\duedate} {December 7, 2010}
\begin{document}
\banner

\section{Unrealistic Memory Scheduling and Perfect Branch Prediction}
\begin{enumerate}[a)]
\item Once we exceed a 32-entry ROB but before we reach a 64-entry ROB, our
  idealized processor exceeds an average uIPC of 4.
\item In order to achieve 95\% of the performance of the 1024-entry ROB
  ($0.95\times 7.33=6.9728$) an ROB must have just over 256 entries.
\end{enumerate}

\section{Conservative Memory Scheduling and Perfect Branch Prediction}
\begin{enumerate}[a)]
\item In our conservative memory scheduling model, having at least a 64-entry
  ROB is required to exceed an average uIPC of 4.
\item In order to achieve 95\% of the performance of the 1024-entry ROB,
  ($0.95\times 4.86=4.61$) an 256-entry ROB will suffice.
\item The largest difference between unrealistic and conversative memory
  happens when both processors have reached their peak uIPC.  At this point,
  unrealistic memory scheduling is 50\% faster (uIPC of 7.33 vs. 4.86).
\end{enumerate}

\section{Perfect Memory Scheduling and Perfect Branch Prediction}

\begin{enumerate}[a)]
\item The ROB must have more than 32 entires to achieve an uIPC greater than 4.
\item In order to achieve 95\% of the performance of the 1024-entry ROB,
  ($0.95\times 7.326=6.95$) having more than a 512-entry ROB will suffice.
\item The largest uIPC difference is with a 128-entry ROB.  At this point,
  unrealistic memory scheduling is 5\% faster than perfect memory
  scheduling (uIPC of 6.39 vs 6.04).
\end{enumerate}

\section{Realistic Branch Prediction}
\begin{enumerate}[a)]
\item The processor never achieves an average uIPC of 4, regardless of ROB
  size.  Making an infinite-entry ROB would likely make little difference, as
  our data shows that there is no performance difference between large
  (i.e. 256 and above) entry ROBs.
\item In order to achieve 95\% of the performance of the 1024-entry ROB,
  ($0.95\times 3.72=3.53$) an ROB must have 128 entries or more.
\item The largest difference between perfect and realistic branch prediction
  occurs when there are 1024 or more entries in the ROB.  When this happens,
  perfect branch prediction is 96\% faster than realistic branch prediction
  (uIPC of 7.326 vs 3.726).

\end{enumerate}

\section{Comparison to Contemporary Processor}
\begin{enumerate}[a)]
\item It seems like having 128 entries in the i7's ROB would be a bad idea.
  Based on our results, having more of them would result in a higher uIPC.
\item One simplification that we made is the huge number of registers.  This
  means that the larger we make the ROB, the more renaming we can do and the
  higher our uIPC is, even with very large ROB sizes.  If we had modelled a
  smaller/more realistic register file, we would have found that the benefits
  of increasing the ROB size stopped sooner, because we would run out of
  physical registers to do renaming.

  Another relevant simplification that we made was that we have enough entries
  in the instruction queue and that we can always fetch N instructions.  If
  these assumptions are not met, then increasing the ROB size will not make any
  performance difference, because instructions will not be able to be processed
  any faster.
\end{enumerate}

\section{Average Cost per Branch Mis-prediction}
\begin{enumerate}[a)]
\item There are three components to the branch misprediction.  One is when a
  mis-prediction occurs, the ROB needs to be flushed and it takes time to
  repopulate this in order to achieve a decent commit throughput.  Another is
  the amount of time that we have spent executing instructions that will not be
  committed.  The third is the fixed cost to redirect the fetch.
\item The cost of a branch prediction consists of the 4 cycle delay that occurs
  after its execution and the amount of time inbetween where it is fetched and
  issued.  The first component is fixed, the second one is a funciton of number
  of entries in the ROB.  So, the larger the ROB, the greater the cost of a
  branch prediction.
\end{enumerate}

\newpage
\section{Graphs}
\graph{Graph A (Questions 1-4)}{figs/q1234.pdf}
\graph{Graph B (Question 5)}{figs/q5.pdf}

\end{document}

