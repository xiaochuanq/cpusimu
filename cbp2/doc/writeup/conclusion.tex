\section{Related work}
Our work mainly depends on Andre Seznec's design and implementation of the O-GEHL predictor, which he submitted to the the first Championship Branch Prediction competition.\cite{seznec2005analysis}\cite{cbp1}\cite{ogehl} In \cite{seznec2005analysis}, he credits research done over a 15 year period for his design.  Among these are the use of combining the results from multiple predictors, which was first introduced in McFarling's seminal work.\cite{mcfarling1993combining}

\section {Conclusion}
In this report, we described our implementations of GEHL and O-GEHL branch predictor, and evaluated the performance compared with baseline branch predictors. Significant performance improvement acheived by GEHL and O-GEHL over 8 distinct long traces. Also, by exploring four important parameters used in GEHL and O-GEHL predictor, we found the best value of threshold $\theta$ varies from trace to trace, and the best value of history length growth rate $\alpha$ is 2.0. The most cost-effective number of tables is 6-bit and the best size of saturating counters is 4-bit. These findings are consistent with Seznec's results. Adaptive history length and adaptive threshold also improved performance during our experiments, but not as signifcant as suggested by previous researches. We also ensured that GEHL and  OGEHL can be pipelined ahead to increase the IPC. Overall, GEHL and OGEHL performs better than baseline predictors in our experiments, yet more complicated features, such as the indexing algorithm should be considered carefully in real implementation.
